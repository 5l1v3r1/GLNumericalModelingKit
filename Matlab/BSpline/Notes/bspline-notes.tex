\documentclass[11pt, oneside]{article}   	% use "amsart" instead of "article" for AMSLaTeX format
\usepackage{geometry}                		% See geometry.pdf to learn the layout options. There are lots.
\geometry{letterpaper}                   		% ... or a4paper or a5paper or ... 
%\geometry{landscape}                		% Activate for rotated page geometry
\usepackage[parfill]{parskip}    		% Activate to begin paragraphs with an empty line rather than an indent
\usepackage{graphicx}				% Use pdf, png, jpg, or eps§ with pdflatex; use eps in DVI mode
								% TeX will automatically convert eps --> pdf in pdflatex		
\usepackage{amssymb}
\usepackage{amsmath}

%SetFonts

%SetFonts


\title{Notes on implementing b-splines}
\author{Jeffrey J. Early}
%\date{}							% Activate to display a given date or no date

\begin{document}
\maketitle
%\section{}
%\subsection{}

This is an attempt to clean-up the notation in Carl de Boor's b-spline notes,
The $m$-th B-spline of order $K=1$ is defined as,
\begin{equation}
X^1_m(t) \equiv \begin{cases}
1      & \text{if $ \tau_m \leq t < \tau_{m+1}$}, \\
0     & \text{otherwise}.
\end{cases}
\end{equation}

All higher order B-splines are defined by recursion,
\begin{equation}
X^K_m(t) \equiv \frac{t - t_m}{t_{m+K-1} - t_m} X^{K-1}_m(t) + \frac{t_{m+K}-t}{t_{m+K} - t_{m+1}} X^{K-1}_{m+1}(t).
\end{equation}
and a path is represented as $x(t) \equiv  X^K_m(t) \xi^m$ where $\xi^m$ are the coefficients.

The $j$-th derivative of this path is,
\begin{equation}
\left(\frac{d}{dt}\right)^j \left(X^K_m(t) \xi^m\right) = \xi_{j+1}^m X^{K-j}_m(t)
\end{equation}
where 
\begin{equation}
\xi_{j+1}^m \equiv \begin{cases}
\xi^m      & \text{for $j=0$}, \\
\frac{\xi_j^m - \xi_j^{m-1}}{ (t_{m+K-j} - t_m)/(K-j) }    & \text{otherwise}.
\end{cases}
\end{equation}
So you compute the coefficients of the higher order derivatives, by differencing the coefficients.

So, de Boor's algorithm is the following sum over $K-j$ non-zero splines for position $t$,
\begin{align}
x^{(j)}(t) =& \sum_{m=1}^{K-j} \xi_{j+1}^m X^{K-j}_m(t) \\
=&  \sum_{m=1}^{K-j} \xi_{j+1}^m \left[ \frac{t - t_m}{t_{m+K-1} - t_m} X^{K-j-1}_m(t) + \frac{t_{m+K}-t}{t_{m+K} - t_{m+1}} X^{K-j-1}_{m+1}(t) \right] \\
=& \sum_{m=1}^{K-j} \xi_{j+1}^m \left[ \frac{t - t_m}{t_{m+K-1} - t_m} X^{K-j-1}_m(t) \right] +  \sum_{m=0}^{K-j-1} \xi_{j+1}^{m-1} \left[ \frac{t_{m+K-1}-t}{t_{m+K-1} - t_{m}} X^{K-j-1}_{m}(t) \right] 
\end{align}
One needs to think of the sum as going from $m=m_i$ to $m=m_i + K-j$ where $m_i$ is the lowest non-zero spline at that point. Thus, there is no reason one can't add a few other splines that are zero at that point. So, it's perfectly reasonable to rewrite the sum as,
\begin{align}
x^{(j)}(t) =& \sum_{m=m_i-1}^{m_i+K-1-j}  \left[ \frac{\xi_{j+1}^m(t - t_m) + \xi_{j+1}^{m-1} (t_{m+K-1}-t)  }{t_{m+K-1} - t_m} \right]  X^{K-j-1}_m(t)
\end{align}
So we just expressed the value of $x^{(j)}(t)$ as a sum of lower order splines (one order less).

\end{document}  